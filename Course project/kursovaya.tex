
\documentclass[12pt,a4paper]{article}
\usepackage{cmap}
\usepackage{mathtext}
\usepackage[T2A]{fontenc}
\usepackage[utf8]{inputenc}
\usepackage[russian]{babel}
\usepackage{amsmath, amsfonts, amssymb}
\usepackage{graphicx}
\usepackage{geometry}
\geometry{left=2cm,right=2cm,top=2cm,bottom=2cm}

\begin{document}

\begin{center}
\Large \textbf{Курсовая работа} \\
\normalsize \textbf{Двухэтапные и многоэтапные задачи стохастического программирования}
\end{center}

\section*{Введение}
Рассматривается задача управления запасами на складе в условиях детерминированного и стохастического спроса. Цель --- минимизация затрат на пополнение и хранение запасов.

\section*{Исходные данные}
\begin{itemize}
    \item Число периодов: 8
    \item Расход продукции: $d = [50, 150, 50, 100, 50, 150, 50, 150]$
    \item Вместимость склада: 400 единиц
    \item Партии заказа: кратные 50 единицам
    \item Функции затрат на хранение $\varphi(t)$ и пополнение $\psi(t)$ заданы таблично
\end{itemize}

\section*{Формализация задачи}
Состояние: уровень запасов $y_k$ в начале периода $k$.\\
Управление: объем заказа $u_k$.\\
Переход: 
\[\
y_{k+1} = y_k + u_k - d_k
\]
Стоимость: сумма затрат на пополнение и хранение запасов.

\section*{Метод решения}
Применяется метод динамического программирования: обратный рекурсивный поиск от последнего периода к первому.

\section*{Решение задачи}

\subsection*{1. Детеминированная задача}
На складе в конце планирования должно быть 0 единиц продукции. Для каждого начального уровня запасов $y_0 = 0, 50, 100, \ldots, 400$ определяется оптимальный план пополнений.

\subsection*{2. Учёт штрафа за остаток}
Если к концу остаётся продукция, за каждые 50 единиц остатка начисляется штраф:
\[\
5 \left( \left\lfloor \frac{l}{3} + 1 \right\rfloor \right) = 5 \times 1 = 5 \text{ денежных единиц}
\]
Расчёты аналогичны первому случаю, но при наличии остатка добавляется штраф.

\subsection*{3. Стохастическая задача}
Учитывается стохастический спрос с заданными вероятностями. Требуется минимизировать математическое ожидание суммарных затрат:
\[\
\mathbb{E}[\text{затраты}] = \sum (\text{вероятность спроса}) \times (\text{затраты при данном спросе})
\]
Переходы вероятностные, так как спрос в каждом периоде случайный.

\section*{Выводы}
Проведённый анализ позволяет определить оптимальные стратегии пополнения запасов в различных условиях: детерминированном, с учётом штрафов и в стохастическом случае. Использование методов динамического программирования обеспечивает получение решений минимизирующих суммарные издержки.

\end{document}
